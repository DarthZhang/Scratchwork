%!TEX root = main.tex
In the landscape of natural language processing (NLP) research, 
dialogue systems used to model human-robot conversations can typically be decomposed into two types: 
task-oriented and chi-chat dialogue systems. 
Chit-chat systems aim to provide reasonable responses to humans with no particular goal in mind; 
conversations are typically open domain. 
On the other hand, task-oriented conversations typically aim to provide specific answers to queries toward the achievement of 
task-specific goals. Examples may include finding products, booking hotels or scheduling problems. 

However, for many real life applications of dialogue modeling, 
conversations often involve a mixture of open-domain and task-oriented dialogue. 
Additionally, human-agent dialogue systems may be evaluated for utility in off-conversation 
purposes rather than accuracy of dialogue generation itself. For example, rather than conversing 
with a user to provide task-specific answers to queries, a chat-bot may be interested in creating 
chit-chat to glean information about the user for off-conversation purposes. 
In this paper, we show how dialogue systems can be used to aid off-conversation tasks. 
In particular, we show how a dialogue management system can be trained from human-human dialogue to generate human-agent dialogue. 
We then show how agent-generated dialogues can be used to extract new knowledge for 
the task-of-interest well as provide data augmentation of the limited conversational data to improve predictive modeling.
