%!TEX root = main.tex

We evaluate our dialogue system on the binary classification task of identifying mild cognitive impairment (MCI) 
among clinical patients. This was doneusing utterance data from provider-patient interviews.
Data was obtained from a randomized controlled behavioral clinical trial conducted at Oregon Health & Sciences University (OHSU) 
to ascertain the effect of unstructured conversation on Alzheimer Disease progression (ClinicalTrials.gov: NCT01571427). 
Conversations are in interviewer-patient, Q&A format. 
The patient responses are unstructured while interviewer questions range over preset question topics. 
There are 14 positive labels (MCI) and 27 negative labels (NL) for the prediction task. 
Figure 2 shows a schematic diagram of our experimental design.


\subsection { Results of Baseline Predictions }
Table 1 shows the performance of various classifiers using averaged skip-thought vector embeddings per patient. 
As a reference, performance using raw word count distributions is also included. 
Similar to Dodge et al., linear SVM with l1-norm penalty achieved the highest performance with 
80.0% AUC over 10-fold validation. 


\subsection { Evaluation of User Simulators }
\subsection { Top-Performing RL-Agents }
\subsection { Performance of Classifiers on Agent-Generated Dialogue}

