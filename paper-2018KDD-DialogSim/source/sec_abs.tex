%!TEX root = main.tex

Dialogue modeling is classically used to generate task-oriented responses to user 
query or to generate open-domain conversations with user. In this paper, we train a human-agent dialogue systems 
by observing only human-human responses in conversations. We show how such a dialogue system can be used for 
data augmentation and performance improvement in off-conversation prediction tasks. 
We propose a framework which contains 4 main components: (1) skip-thought embedding of response utterances, 
(2) training customized user-simulator for each subject, 
(3) training a reinforcement learning agent to learn a unified action-value function for 
query-answer responses for the task of interest, and 
(4) using trained agents to generate new dialogue episodes for data augmentation and prediction. 
We evaluate the performance of this framework on classification of mild-cognitive impairment 
(MCI) with utterance data from 41 OHSU patients. 
Our methods achieved 78.6% AUC on 5-fold validation with our current framework, 
an improvement over a previous benchmark study (72.5% AUC with sparse SVM). 
Additionally, our agent learned individualized policies for producing query 
sequences which can minimize the number of turns required to make accurate diagnostic predictions.
